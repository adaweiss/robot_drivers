\documentclass[a4paper,12pt]{article}
\usepackage{polski}
\usepackage[T1]{fontenc}
\usepackage[utf8]{inputenc}
\usepackage[top=2cm, bottom=2cm, left=3cm, right=3cm]{geometry}
\usepackage{indentfirst}
\usepackage{enumerate}

\makeatletter
\newcommand{\linia}{\rule{\linewidth}{0.4mm}}
\renewcommand{\maketitle}{\begin{titlepage}  
    \vspace*{1cm}
    \begin{center}
  Sterowniki robotów - projekt\\
Termin zajęć\\
\textit{środa TP 11:15}
    \end{center}
      \vspace{3cm}
    \begin{center}
     \LARGE \textsc {\@title}
         \end{center}
     \vspace{1cm}
    
    \begin{center}
    Autorzy:\\
   \textit{\@author} \\
\vspace{1cm}
Grupa projektowa nr 2\\
\vspace{2cm}
Prowadzący:\\
 \textit{mgr inż. Wojciech Domski}

     \end{center}
      \vspace{1cm}
    
    
    \vspace*{\stretch{6}}
    \begin{center}
    \@date
    \end{center}
  \end{titlepage}
}
\makeatother
\author{Beata Berajter 218629\\
Ada Weiss 218641 }%wpisać indeks
\title{Logger danych.\\Akwizycja danych o temperaturze wraz z przeglądaniem danych archiwalnych na LCD }


\begin{document}
\newpage
\maketitle
\newpage
\tableofcontents

\newpage
\section{Cel projektu}
Celem projektu jest zapoznanie się z programowaniem płytki deweloperskiej STM32L476G-DISCO z wykorzystaniem różnych preyferiów mikrokontrolera.
\section{Opis projektu}
Zamieniamy




\begin{thebibliography}{99}
\bibitem{pa} Dokumentacja płytki deweloperskiej
\emph{http://dl.btc.pl/kamami_wa/STM32L476G-DISCO_USER_MANUAL.pdf}, 

\end{thebibliography}
\end{document}