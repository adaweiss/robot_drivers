\documentclass[a4paper,12pt]{article}
\usepackage{polski}
\usepackage[T1]{fontenc}
\usepackage[utf8]{inputenc}
\usepackage[top=2cm, bottom=2cm, left=3cm, right=3cm]{geometry}
\usepackage{indentfirst}
\usepackage{enumerate}
\usepackage{hyperref}
\makeatletter
\newcommand{\linia}{\rule{\linewidth}{0.4mm}}
\renewcommand{\maketitle}{\begin{titlepage}  
    \vspace*{1cm}
    \begin{center}
  Sterowniki robotów - projekt\\
Termin zajęć\\
\textit{środa TP 11:15}
    \end{center}
      \vspace{3cm}
    \begin{center}
     \LARGE \textsc {\@title}
         \end{center}
     \vspace{1cm}
    
    \begin{center}
    Autorzy:\\
   \textit{\@author} \\
\vspace{1cm}
Grupa projektowa nr 2\\
\vspace{2cm}
Prowadzący:\\
 \textit{mgr inż. Wojciech Domski}

     \end{center}
      \vspace{1cm}
    
    
    \vspace*{\stretch{6}}
    \begin{center}
    \@date
    \end{center}
  \end{titlepage}
}
\makeatother
\author{Beata Berajter 218629\\
Ada Weiss 218641 }%wpisać indeks
\title{Logger danych.\\Akwizycja danych o temperaturze wraz z przeglądaniem danych archiwalnych na wyświetlaczu LCD }


\begin{document}
\newpage
\maketitle
\newpage
\tableofcontents

\newpage
\section{Cel projektu}
Celem projektu jest zapoznanie się z programowaniem płytki deweloperskiej STM32L476G-DISCO oraz wykorzystanie różnych peryferiów mikrokontrolera.
\section{Opis projektu}

Zadanie polega na tym, aby odczytywać dane z wewnętrznego termometru w określonych odstępach czasowych
jednoczenie odczytując czas z RTC\\
Następnie dane te mają być przesyłane przez Quad SPI do pamięci zewnętrznej Flash. Zapamiętywanych jest 100 ostatnich wyników. Pamięć obsługiwana ma być w taki sposób, aby:
\begin{itemize}
    \item zapisywać ostatnie miejsce w pamięci, w które został wpisany pomiar,
    \item w przypadku przekroczenia dostępnego miejsca - wyniki kolejno nadpisywały się w pamięci.
\end{itemize}

Ostatnią częścią projektu będzie obsługa wyświetlacza LCD umożliwiająca za pomocą joysticka przejrzenie wszystkich zapisanych w pamięci pomiarów.

\begin{thebibliography}{99}
\bibitem{pa} Dokumentacja płytki deweloperskiej 
\url {http://www.st.com/content/ccc/resource/technical/document/user_manual/d1/84/86/4b/08/82/47/91/DM00172179.pdf/files/DM00172179.pdf/jcr:content/translations/en.DM00172179.pdf}
\bibitem{pa} Application note. Quad-SPI (QSPI) interface on STM32 microcontrollers
\url {http://www.st.com/content/ccc/resource/technical/document/application_note/group0/b0/7e/46/a8/5e/c1/48/01/DM00227538/files/DM00227538.pdf/jcr:content/translations/en.DM00227538.pdf}

\end{thebibliography}
\end{document}